\section{Instantiable Transition Systems} \label{its-def}

 In this section we formally define Instantiable Petri Nets (IPN).
While simple, we will show in the next sections that IPN allow on
one hand to adequately capture important concepts of high level
modeling languages such as UML, and on the other hand to enable
powerful model-checking techniques.

\subsection{IPN Definition}
\label{ss:ipndef}

Instantiable Petri Nets are built upon the standard Petri net
formalism semantics. However IPN introduce the notion of net type
 and net instance as well as a composition mechanism based solely on
transition \emph{synchronization} (no place or transition fusions).

An \textbf{IPN type} is a tuple
$type=\tuple{S,\textit{LabeledStates},T,V,\mathit{Enable},\mathit{Fire}}$:
\begin{itemize}
\item $S$ is a set of states;
\item $\textit{LabeledStates} \subseteq S$ is a finite subset of states which bear a label;
\item $T$ is a finite set of \emph{transitions};
\item $V: T \rightarrow \{$public,private$\}$ is a function assigning a \emph{visibility} to each transition;
\item $\mathit{Enable} : S \times \bag(T) \setminus \emptyset \rightarrow \mathds{B}$ is the enabling function. The notation $\bag(A)$ denotes a multiset over a set $A$. For a multiset $\tau \in \bag(T)$ and a transition $t$, we denote by $\tau(t)$ the number of occurrences of $t$ in $\tau$. A non-empty multiset of transitions $\tau \in \bag(T)$ is said to be \emph{simultaneously enabled} in state $s$ if $\mathit{Enable}(s,\tau) = \mathit{true}$;
\item $\mathit{Fire} : S \times T \rightarrow S$ is the transition function.
\end{itemize}

In the following we note $tuple.X, tuple.Y \ldots$ the element $X$
(resp. $Y$\ldots) of a tuple $tuple$. Let $\mathit{Types}$ denote a
set of IPN types. An \textbf{IPN instance} $i$ is defined by its IPN
type, noted $\mathit{type}(i) \in \mathit{Types}$.
%We further denote
%$\mathit{events}(i)$ the set of \emph{public} transitions of
%$\mathit{type}(i)$, i.e. $\mathit{events}(i) = \{t \in
%\mathit{type}(i).T \mid \mathit{type}(i).V(t) = \mathit{public}\}$.
 An IPN instance $i$ may be associated to a state $s \in
\mathit{type}(i).S$. We will use in the following the terminology:
``assign a state $s$ to instance $i$''.

The IPN definition is extensible by providing new sorts of types that
respect the definition of IPN types given above.  In practice, two
sorts of types are defined:

A \textbf{unitary IPN type} is a tuple
$\tuple{P,T,\mathit{Pre},\mathit{Post},M,\textit{LabeledStates},V,\mathit{Enable},\mathit{Fire}}$:

\begin{itemize}
\item $P$ is a finite set of places;
\item $T$ is a finite set of \emph{elementary transitions};
\item $\mathit{Pre}$ and $\mathit{Post}$ : $P \times T \rightarrow
\bbbn$ are the pre and post functions labeling the arcs;
\item $M \equiv S$ is a set of states called markings, such that each marking $m$ of $M$ is of the form $m : P \rightarrow \nat$;
\item $\textit{LabeledStates} \subseteq M$ is a finite subset of markings which bear a label;
\item $V: T \rightarrow \{$public,private$\}$ is a function assigning a visibility to each transition;
\item $\mathit{Enable}$: a multiset of transitions $\tau$ is simultaneously enabled in a marking $m$
 if for each place $p$, the condition $\sum_{t \in T} \tau(t)\cdot \mathit{Pre}(p,t) \leq m(p)$ holds;
\item $\mathit{Fire}$:  the firing of a transition $t$ in a marking $m$ leads to a marking $m'=\mathit{Fire}(t,m)$ defined
 by $\forall p \in P, m'(p) = m(p) - \mathit{Pre}(p,t) + \mathit{Post}(p,t)$.
\end{itemize}

A \textbf{composite IPN type} is a tuple  $C=\tuple{I,
\mathit{Sync}, S,\textit{LabeledStates},
V,\mathit{Enable},\mathit{Fire}}$:
\begin{itemize}
\item $I$ is a finite set of IPN instances, said to \emph{be contained} by $C$. We further require that the type of each IPN instance $i \in I$ preexists when defining these instances, in order to prevent circular or recursive type definitions.
\item $\mathit{Sync} \equiv T$ is a finite set of transitions called synchronizations.
Let $\mathit{Part}$ be the set of pairs $\{\tuple{i,e}\mid i \in I
\wedge e \in \{t \in \mathit{type}(i).T \mid \mathit{type}(i).V(t) =
\mathit{public}\}$. A synchronization $s$ is defined as a multiset
over $\mathit{Part}$.
 For a given instance $i \in I$, we say that the parts of $s$ of the form $\tuple{i,e_j}$ \emph{belong to} instance $i$.
{\color{red} La notation $s$ a deja ete utilise pour state. D'ou
vient j ?}
\item $CS \equiv S$ is a set of composite states, such that a state $cs \in CS$ assigns to each instance $i \in I$ a state $cs(i) \in type(i).S$
\item $\textit{LabeledStates} \subseteq CS$ is a finite subset of states which bear a label, where $cs \in \textit{LabeledStates} \implies \forall i \in I, cs(i) \in type(i).\textit{LabeledStates}$
\item $V : \mathit{Sync} \rightarrow \{$public,private$\}$ is a function assigning a visibility to each synchronization
\item $\mathit{Enable}$: A non-empty multiset of synchronizations $\sigma \subseteq \mathit{Sync}$ is simultaneously enabled in a composite state $cs$ if  $\forall i \in I$, the submultiset of parts $\tuple{i.e_j}$ of $\sigma$ that belong to instance $i$ are simultaneously enabled in state $cs(i)$
\item $\mathit{Fire}$: the firing of a synchronization $s \in \mathit{Sync}$ on a state $cs$ yields a state $cs'=\mathit{Fire}(cs,s)$ such that $\forall i \in I$, $cs'(i)$ is obtained by firing in sequence on $cs(i)$ all the parts of $s$ that belong to $i$.
\end{itemize}

An \textbf{\textit{IPN model}} is a tuple $\tuple{ \mathit{Types},
\mathit{main}, s_0}$:
\begin{itemize}
\item $\mathit{Types}$ is a set of IPN type definitions,
\item $\mathit{main}$ is an \textit{IPN instance} such that $\mathit{type(main)} \in \mathit{Types}$, that defines the container of the model,
\item $s_0 \in \mathit{type(main).LabeledStates}$ is the initial state assigned to the ``main'' instance.
\end{itemize}


\textbf{IPN Semantics.}
Enabling conditions are defined to take into account the notion of
transition visibility: for an IPN instance $i$ in state $s$ a private
transition $t \in \mathit{type}(i).T$ is enabled if $Enabled(s,\{t\})$ holds. Firing such an enabled transition $t$ from a state $s$ yields a new
state $s'=\mathit{type(i).Fire}(s,t)$.  If the type of
$i$ is a composite type, private transitions of contained instances
$\mathit{type}(i).I$ of $i$ are themselves potentially enabled and may be fired according to the
same schema.  Public transitions are thus only fireable through an external
synchronization, and not in isolation (i.e. locally to an instance).

IPN thus have simple semantics, strongly based on the notion of transition
synchronization.  Limiting the composition semantics in this way is
favorable to using various verification algorithms that exploit
compositional verification \cite{}.

\textbf{IPN Expressiveness.} In terms of expressiveness, IPN are
equivalent to standard Petri nets. Indeed, IPN can be ``flattened''
to an equivalent Petri net using a simple algorithm.  The algorithm
simply consists in creating places in the flattened net using a
fully qualified naming scheme : e.g. $\mathit{main}.i1.i2.p1$ for
the place $p1$ of the IPN instance $i1$ of the composite IPN
instance $i2$ which is contained by the $\mathit{main}$ instance.
Similarly, a transition is created for each private transition of
the ``main'' instance, by fusing the definition of its parts.
Additional transitions are added for each private transition of
instances contained in the ``main'' instance. This algorithm is
applied recursively to contained instances until unitary net types
are reached.

\textbf{IPN Conciseness.} However, we can note that IPN are potentially
\emph{exponentially} more concise than Petri nets. Indeed, the flattening
procedure duplicates transitions for each instance of a net type.  Consider
an IPN instance of $M_n$, where the type $M_n$ is recursively
built as a composition of two instances of a net type $M_{n-1}$. For
each place (resp. transition) of the unitary IPN $M_0$, $2^n$ places
(resp. transitions) will be produced in the flattened
representation. However, the IPN representation of the model stays
linear in size with respect to $n$.
