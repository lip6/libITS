\section{Model-checking of IPN}
\label{ipn-verif}

%The previous section has shown how IPN are a formalism well adapted to
%capturing some important concepts of high level specification
%languages. 
We show in this section how the compositional definition of
the specification and the identification of similarities through
multiple instantiations of a given type can be exploited to within
model-checking algorithms to fight the combinatorial state-space
explosion problem.

\subsection{Set Decision Diagrams}
\label{sub:set_decision_diagrams}

In this subsection we recall the salient points of Hierarchical Set
Decision Diagrams, a data structure based on the principles of
decision diagram technology (node uniqueness thanks to a canonical
representation, transition relation acting on a \emph{set} of states,
dynamic programming, ordering issues \ldots). They feature two main original
aspects: $a)$ the support of hierarchy in the
representation and $b)$ the definition of user operations through a
mechanism called \emph{inductive homomorphisms} (instead of more
decision diagrams) which gives unprecedented freedom and flexibility
to the user.

Hierarchical Set Decision Diagrams (SDD) defined in \cite{CT05a}, are
shared decision diagrams in which arcs of the structure are labeled by
a \emph{set} of values, instead of a single valuation.  This set may
itself be represented by an SDD, thus when labels are SDD, we think of
them as hierarchical decision diagrams. This definition is taken
practically verbatim from \cite{atpn08saturation} where it was adapted
for more clarity from \cite{CT05a}.

SDD are data structures for representing sequences of assignments of
the form $\omega_1 \in s_1 ;\omega_2 \in s_2; \cdots \omega_n \in s_n$
where $\omega_i$ are variables and $s_i$ are sets of values.

We assume no variable ordering, and the same variable can occur
several times in an assignment sequence. We define the usual
terminal $1$ to represent accepting sequences. The terminal $0$ is
also introduced and represents the empty set of assignment
sequences. In the following, $Var$ denotes a set of variables, and
for any $\omega$ in $Var$, $\Dom(\omega)$ represents the domain of
$\omega$ \emph{which may be infinite}.

\begin{definition}[Set Decision Diagram]\label{SDD_Def}
The set $\SDD$ of SDD is inductively defined by $\de\in \SDD$ if:
\begin{itemize}
\item $\de \in \{0,1\}$ or
\item $\de=\tuple{\omega,\pi,\alpha}$ with:
 \begin{itemize}
    \item $\omega \in Var$.
    \item $\pi = s_0 \uplus \cdots \uplus s_n$ is a finite  partition of $\Dom(\omega)$,
    i.e. $\forall i\neq j, s_i\cap s_j=\emptyset, s_i\neq\emptyset, n$ finite.
    \item $\alpha : \pi \rightarrow \SDD$, such that $\forall i \neq j, \alpha(s_i) \neq \alpha(s_j)$.
\end{itemize}
\end{itemize}
% We will simply note $d=(\omega,\alpha)$ the node $(\omega,\pi,\alpha)$ as $\alpha$ implicitly defines $\pi$.
%We denote by $\omega \fireseq{s} \de$, the SDD $(\omega,\pi,\alpha)$
%with $\pi= s \uplus (\Dom(\omega) \setminus s)$,
%$\alpha(s)=\de,\alpha(\Dom(\omega) \setminus s) = 0$. 
By convention,
when it exists, the element of the partition $\pi$ that maps to the
SDD $0$ is not represented.
\end{definition}

Despite its apparent simplicity, this definition supports rich and
complex data:
\begin{itemize}
\item the definition supports domains of infinite size (e.g. $\Dom(e)
= \mathds{R}$), provided that the partition size remains finite
(e.g. $]0..3],]3..+\infty]$). This feature could be used to model
clocks for instance (as in \cite{WangIEEE04}). It also places the
expressive power of SDD above most variants of DD.
\item $\SDD$ or other variants of decision diagrams can be used as
 the domain of variables, introducing hierarchy in the data structure. 
\item SDD can handle paths of variable lengths, if care is taken when
writing operations to avoid creating so-called incompatible
sequences (see \cite{CT05a}). This feature is useful when
representing dynamic structures such as queues.
\end{itemize}

SDD support standard set theoretic operations (union, intersection,
set difference). They also offer a concatenation operation $\de_1
\cdot \de_2$ which replaces $1$ terminals of $\de_1$ by $\de_2$. It
corresponds to a cartesian product. In addition, inductive homomorphisms are
defined as a powerful and flexible mechanism to define application
specific operations. Unfortunately, due to size constraints,
homomorphisms are not presented here; an interested reader should read
\cite{atpn08saturation} where much more detail is available.

An important recent result is that we have defined a set of rewriting
rules for homomorphisms \cite{atpn08saturation}, allowing to
automatically make use of the decision diagram saturation algorithms
originally due to Ciardo \cite{ciardo03saturation}; when computing the
fixpoint of a transition relation over a set of states which is
central to the model-checking problem, this algorithm offer gains of
one to three orders of magnitude over classical BFS fixpoint
algorithms.

Hierarchical Set Decision Diagrams are available within
\texttt{libddd} a C++ library freely available under the terms of
GNU LGPL, at \url{http://ddd.lip6.fr}.

\subsection{Exploiting IPN Structure}
\label{sub:perfs}

We compare in this section the performance of our SDD based
model-checking engine when producing the full state-space of a system on
the flattened representation of an IPN to a version that uses a state
encoding that follows the hierarchical structure of the IPN
specification. All the performance runs exploit a saturation algorithm,
which accounts for the extremely good overall performance.

\begin{table}[t]
%@{\,\vrule width 5pt\,}
	% \renewcommand{\arraystretch}{0.9}
	
	\begin{center}
	\begin{tabular}{|c|c|>{\columncolor[gray]{.9}}c|
		>{\columncolor[gray]{.9}}c||c|c|c||c|c|>{\footnotesize}c|}
	
	\cline{3-10}
	\multicolumn{2}{c|}{} &
	\multicolumn{2}{>{\columncolor[gray]{.9}}c||}{Final} &
%	\multicolumn{3}{c||}{Hierarchical} &
	\multicolumn{3}{c||}{Flat} &
	\multicolumn{3}{c|}{IPN} \\	
	
%	\multicolumn{2}{c|}{} &
%	\multicolumn{2}{>{\columncolor[gray]{.9}}c||}{\#} &
%	\multicolumn{3}{c||}{\emph{Chaining Loop}} &
%	\multicolumn{3}{c||}{\emph{Automatic Sat.}} &
%	\multicolumn{3}{c|}{\emph{Automatic Sat.}} \\	

	\hline
	Model & States
		& Flat & IPN 					% Nb. Final nodes
%		& T. & Mem. & Peak		% Hierarchical Chaining Loop
		& T. & Mem. & Peak		% Flat Automatic Saturation
		& T. & Mem. & Peak \\		% Hierarchical Automatic Saturation

	Size & \#
		& \#  & \#
%		& (s) & (MB) & \#
		& (s) & (MB) & \#
		& (s) & (MB) & \#  \\
		
	\hline
	
	%%%%%%%%%%%%%%%%%%%%%%%%%%%%%%%%%%%%%%%%%%%%%%%%%%%%%%%%
	
	\hline
	\hline
	\multicolumn{10}{|c|}{LOTOS Specification}\\

	\hline
		& 9.8e+21
		& -- & 1085
%		& -- & -- & --
		& -- & -- & --
		& 1.47 & 74.0 & 110e+3 \\
	
	\hline

	%%%%%%%%%%%%%%%%%%%%%%%%%%%%%%%%%%%%%%%%%%%%%%%%%%%%%%%%
	
	\hline
	\hline
	\multicolumn{10}{|c|}{Dining Philosophers}\\

	\hline
	100 & 4.9e+62
		& 2792 & 419 					% Nb. Final nodes
%		&  1.9 & 112 & 276e+3		% Hierarchical Chaining Loop
		&  0.2 & 20 & 18040		% Flat Automatic Saturation
		&  0.07 & 5.2 & 4614 \\		% Hierarchical Automatic Saturation

	\hline
	200 & 2.5e+125
		& 5589 & 819 					% Nb. Final nodes
%		& 7.9 & 446 & 1.1e+6		% Hierarchical Chaining Loop
		& 0.7 & 58.1 & 36241		% Flat Automatic Saturation
		& 0.2 & 10.6 & 9216 \\		% Hierarchical Automatic Saturation
	
	\hline
	1000 & 9.2e+626
		& 27989 & 4019 					% Nb. Final nodes
%		& -- & -- & --		% Hierarchical Chaining Loop
		& 14 & 1108 & 182e+3		% Flat Automatic Saturation
		& 4.3 & 115 & 46015 \\		% Hierarchical Automatic Saturation
	
 	\hline
 	4000 & 7e+2507
		& -- & 16019				% Nb. Final nodes
%		& -- & -- & --		% Hierarchical Chaining Loop
		& -- & -- & --		% Flat Automatic Saturation
		& 77 & 1488 & 184e+3 \\		% Hierarchical Automatic Saturation
	
		
	\hline

	%%%%%%%%%%%%%%%%%%%%%%%%%%%%%%%%%%%%%%%%%%%%%%%%%%%%%%%%
	
	\hline
	\hline
	\multicolumn{10}{|c|}{Slotted Ring Protocol}\\

	\hline
	10 & 8.3e+09
		& 1283 & 105 					% Nb. Final nodes
%		& 1.1 & 48 & 90043		% Hierarchical Chaining Loop
		& 0.2 & 16 & 31501		% Flat Automatic Saturation
		& 0.03 & 3.5 & 3743 \\		% Hierarchical Automatic Saturation

	\hline
	50 & 1.7e+52
		& 29403 & 1345 					% Nb. Final nodes
%		& -- & -- & --		% Hierarchical Chaining Loop
		& 22 & 1054 & 2.4e+6  		% Flat Automatic Saturation
		& 5.1 & 209 & 461e+3 \\		% Hierarchical Automatic Saturation
	
	\hline
	100 & 2.6e+105
		& -- & 5145 					% Nb. Final nodes
%		& -- & -- & --		% Hierarchical Chaining Loop
		& -- & -- & --		% Flat Automatic Saturation
		& 22 & 816 & 1.7e+6 \\		% Hierarchical Automatic Saturation
	
	\hline
	150 & 4.5e+158
		& -- & 11445 					% Nb. Final nodes
%		& -- & -- & --		% Hierarchical Chaining Loop
		& -- & -- & --		% Flat Automatic Saturation
		& 60 & 2466 & 5.6e+6 \\		% Hierarchical Automatic Saturation
	

	\hline

	%%%%%%%%%%%%%%%%%%%%%%%%%%%%%%%%%%%%%%%%%%%%%%%%%%%%%%%%
	
	\hline
	\hline
	\multicolumn{10}{|c|}{Kanban}\\

	\hline
	100 & 1.7e+19
		& 11419 & 511 					% Nb. Final nodes
%		& 12 & 145 & 264e+3		% Hierarchical Chaining Loop
		& 2.9 & 132 & 309e+3		% Flat Automatic Saturation
		& 0.4 & 11 & 14817 \\		% Hierarchical Automatic Saturation
	
	\hline
	200 & 3.2e+22
		& 42819 & 1011 					% Nb. Final nodes
%		& 96 & 563 & 1e+6		% Hierarchical Chaining Loop
		& 19 & 809 & 1.9e+6		% Flat Automatic Saturation
		& 2.2 & 37 & 46617 \\		% Hierarchical Automatic Saturation
	
	
 	\hline
 	300 & 2.6e+24
 		& 94219 & 1511 					% Nb. Final nodes
%		& -- & -- & --		% Hierarchical Chaining Loop
 		& 60 & 2482 & 5.7e+6		% Flat Automatic Saturation
 		& 7 & 78 & 104e+3 \\		% Hierarchical Automatic Saturation

	\hline
	700 & 2.8+28
		& -- & 3511 					% Nb. Final nodes
%		& -- & -- & --		% Hierarchical Chaining Loop
		& -- & -- & --		% Flat Automatic Saturation
		& 95 & 397 & 523e+3 \\		% Hierarchical Automatic Saturation


	\hline

	%%%%%%%%%%%%%%%%%%%%%%%%%%%%%%%%%%%%%%%%%%%%%%%%%%%%%%%%
	
	\hline
	\hline
	\multicolumn{10}{|c|}{Flexible Manufacturing System}\\

	\hline
	50 & 4.2e+17
		& 8822 & 917 					% Nb. Final nodes
%		& 13 & 430 & 530e+3		% Hierarchical Chaining Loop
		& 2.7 & 105 & 222e+3		% Flat Automatic Saturation
		& 0.4 & 16 & 23287 \\		% Hierarchical Automatic Saturation
	
	\hline
	100 & 2.7e+21
		& 32622 & 1817 					% Nb. Final nodes
%		& -- & -- & --		% Hierarchical Chaining Loop
		& 19 & 627 & 1.3e+6		% Flat Automatic Saturation
		& 1.9 & 50 & 76587 \\		% Hierarchical Automatic Saturation

	\hline
	150 & 4.8e+23
		& 71422 & 2717 					% Nb. Final nodes
%		& -- & -- & --		% Hierarchical Chaining Loop
		& 62 & 1875 & 3.8e+6		% Flat Automatic Saturation
		& 5.3 & 105 & 160e+3 \\		% Hierarchical Automatic Saturation
	
	% \hline
	% 200 & States
	% 	& -- & 3617 					% Nb. Final nodes
	% 	& -- & -- & --		% Hierarchical Chaining Loop
	% 	& -- & -- & --		% Flat Automatic Saturation
	% 	& Time & Mem. & Peak \\		% Hierarchical Automatic Saturation
	
	\hline
	300 & 3.6e+27
		& -- & 5417 					% Nb. Final nodes
%		& -- & -- & --		% Hierarchical Chaining Loop
		& -- & -- & --		% Flat Automatic Saturation
		& 33 & 386 & 590e+3     \\		% Hierarchical Automatic Saturation
	

	\hline
	
	%%%%%%%%%%%%%%%%%%%%%%%%%%%%%%%%%%%%%%%%%%%%%%%%%%%%%%%%

	
	\end{tabular}
	\end{center}
	\caption{Compared performance of IPN against a flat representation}
	\label{tab:performances_automatic_saturation}
\end{table}

The examples chosen here are classical examples from
model-checking benchmark literature (mostly taken from \cite{ciardo03saturation}),
 which are parametric in the number of repeated instances of a subsystem.
 The benchmark includes 4 parameterized models: the
well-known Dining Philosophers and Kanban models, a model of the slotted ring
protocol and a model of a flexible manufacturing system. We have also used as benchmark
a LOTOS specification obtained from a true industrial case-study (it was
generated automatically from a LOTOS specification -- 8,500 lines of LOTOS
code + 3,000 lines of C code -- by Hubert Garavel from INRIA).

``--'' entries indicate that the state space's generation did not
finish because of the exhaustion of the computer's main memory.

The \emph{``Final''} columns show the final number of decision diagram
nodes needed to encode the state spaces for hierarchical IPN and a
flattened representation of the same model.  Clearly, the flattened
version needs at least an order of magnitude more nodes to store a
state space. 

The gains we observe are attributed to the fact that subsystems that
are instances of the same net type share the transition relation cache
and are liable to share the state representation of subcomponents,
when the model exhibits some regularity. This experiment shows that
using a structured specification can help detect similarity of
behavior in parts of a model, thus enabling more compact representations.

\subsection{Impact of IPN depth}
\label{sub:perfs}

We have run experiments to measure the impact of using several levels of depth in the encoding.

We define the two following strategies, when encoding a circular model of size $n$:
\begin{itemize}
\item Recursive($\mathit{grain}$): in this setting, we build a representation with $\mathit{grain}$ variables per level. For instance, with 
$n=100$ elements and $\mathit{grain}=5$, we encode 5 components containing 20 instances, each of them encoded using 5 components containing 4 instances, yielding a depth of 3. So we have 5 variables at level 3, 5 variables at level 2, 4 variables at level 1 and $|places|$ variables representing places at level $0$.
\item Depth2($\mathit{grain}$): in this setting, we build a representation with $n / \mathit{grain}$ variables, and a fixed depth of 2. For instance, with 
$n=100$ elements and $\mathit{grain}=5$, we encode 5 components containing 20 instances. So we have 5 variables at level 2, 20 variables at level 1, $|places|$ variables representing places at level $0$. 
\end{itemize}

The case \emph{Recursive($2$)} yields a binary encoding as presented
in the philosopher example (section~\ref{ss:philo}).  The case
\emph{Depth2($1$)} yields an encoding very similar to Ciardo's MDD
\cite{tacas03saturation}, with $n$ variables each representing the
state of a unitary net. \emph{Depth2($1$)} is equivalent to the measures reported 
in the $IPN$ column of table~\ref{tab:performances_automatic_saturation}.

Table~\ref{tab:depth1} presents the evolution of the \emph{Depth2} strategy on the ring model
for n varying from $1$ to $100$ for a sampling of values of the \emph{grain} parameter. We present 
 the time, memory and final representation size in number of SDD nodes. 
Note that the value given for memory is the peak memory size, and includes the size of caches, so it may not match
 the final node size.


Table~\ref{tab:depthrec} presents the evolution of the \emph{Recusrive} strategy on the ring model
for a sampling of values of the \emph{grain} parameter.  We also present 
 the time, memory and final representation size in number of SDD nodes. 