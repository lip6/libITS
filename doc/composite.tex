\mysubsubsection{Composite ITS.}  We now recall from
\cite{tacas09tmphk} the definition of a composite ITS type, designed
to offer support for the hierarchical composition of ITS instances.

\textbf{Notations:} Given a cartesian product $Z=Z_1\times\cdots\times Z_n$ of
sets $Z_1,\cdots, Z_n$, we denote by $\pi_i$ the projection operator $Z
\mapsto Z_i$. For a set $I=\{i_1,\ldots,i_n\}$ of ITS instances, $S_I$ is the
set $type(i_1).S \times \ldots \times type(i_n).S$ and $A_I$ is the set
$\lang{type(i_1).A} \times \ldots \times \lang{type(i_n).A}$. Cardinality of
$I$ is denoted by $|I|$.

\begin{definition}[Composite]
\label{def:composite}
A \textbf{composite} is a tuple $C=\tuple{I, s_0, \Ts, A, \lab}$ where:
\begin{itemize}
\item $I$ is a finite set of ITS instances, said to be
  \emph{contained} by $C$. We further require that the type of each
  ITS instance preexists when defining I, in order to prevent circular
  or recursive type definitions.
\item $s_0 \in S_I$ is the initial state;
$\Ts \subset A_I$ is the finite set of synchronizations;
$A$ is a set of action labels, which contains the label \labloc{} and
$\lab : \Ts \mapsto A$ is the labeling function
\end{itemize}
\end{definition}


\textbf{Notations:} The next state function $\nexti : S_I \times A_I
\mapsto 2^{S_I}$ is used in definition~\ref{def:composite-its}. It is defined
for $s, s' \in S_I$ and $\sigma \in A_I$ by:

$s' \in \nexti(s,\sigma) \ \mathit{iff} \ \forall i \in I, \pi_i (s') \in
type(i).\succs{}(\pi_i(s), \pi_i (\sigma))$

\begin{definition}[ITS Semantics of a Composite]
\label{def:composite-its}
The ITS type
$\tau = \tuple{S,A,\locals{},\succs{}}$
corresponding to a composite $C=\tuple{I, s_0, \Ts, A', \lambda}$, is defined by:
\begin{itemize}
\item $S = S_I$ ;
%$\istate{} = S_0$; 
$A = A' \setminus \{ \labloc \}$;
\item $\locals: S \mapsto 2^S$ is defined for $s,s' \in S$ by: $s' \in \mathit{Locals}(s)$ iff
$$
\left\{
\begin{array}{cl}
 & \exists i \in I, \pi_i (s') \in \mathit{type}(i).\mathit{Locals}( \pi_i(s)) \land \forall j \in I, j \neq i, \pi_j(s') = \pi_j(s)  \\
or & \exists \sigma \in \Ts, \lab(\sigma) = \labloc, s' \in \nexti(s,\sigma)
\end{array}
\right.
$$

\item $\succs{}: S \times \lang{A} \mapsto 2^S$ is defined for $s,s'
  \in S, \ w=a_1\cdots a_n \in \lang{A}$ by:
  $s' \in \succs{}(s,w)$ iff \\
  $\exists \sigma_1,\ldots,\sigma_n \in \Ts, \exists s_0,\ldots,s_n
  \in S, \ \forall j \in [1..n], \ \lab(\sigma_j) = a_j \land s_j
  \in \nexti(s_{j-1},\sigma_j) \land s_0=s \land s_n=s'$.

\end{itemize}
\end{definition}

Definition \ref{def:composite-its} describes an implementation of the
generic ITS type contract. It contains either elementary instances
(such as LTS, or as we will use later in this paper, a discrete timed
transition system), or recursively other instances of composite
nature.

$\locals{}(s)$ is defined as states reachable from $s$ through the
occurrence of any synchronization associated to the local label
\labloc, or states due to the action of \locals{} in any nested
instance (without affecting the other instances).

$\succs{}(s,w)$ is obtained by composing the effects of each label $a$
in the word $w$ using the cartesian product. Labels associated with a
set composed of a single synchronization are simply concatenated. For
labels that associated with a larger set of synchronizations, only one
of the synchronizations is required to occur when the label occurs.

For instance, let us consider in figure~\ref{fig:sync:tab1} (section~\ref{sec:tpn}) a representation of a composite ITS type. It contains
two instances $t_0$ and $t_1$ of an ITS type ``Train'', and defines $5$
synchronizations (lines in the table) and three labels \textbf{App},
\textbf{Exit} and $\elapse$. A state $s$ of this composite is thus defined as
a cartesian product of the state of instance $t_0$ (noted $\pi_{t_0}(s)$) and
$t_1$. The successors obtained by $\succs{}(s,$ \textbf{App}$)$ are the states
in which either $t_0$ or $t_1$ have fired \textbf{App} and the state of the
other instance is unchanged (e.g. $s'$ such that $\pi_{t_0}(s') \in
Train.\succs{}(\pi_{t_0}(s))$ and $\pi_{t_1}(s') = \pi_{t_1}(s)$ or vice
versa). There is no local ($\labloc$ labeled) synchronization in this
example, thus successors by $\locals{}$ is defined as states in which either 
$t_0$ or $t_1$ have progressed by $\mathit{Train}.\locals$.

