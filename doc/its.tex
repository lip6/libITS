\section{Instantiable Transition Systems}

% \subsection{ITS Concepts}
% \label{sec:ipn-def}

This section summarizes the definition of Instantiable Transition Systems
(ITS)~\cite{tacas09tmphk}. They allow to take advantage of SDD characteristics
for the description of component based systems. ITS describe a minimal Labeled
Transition System (LTS) style formalism using notions of $type$ and $instance$
to emphasize locality of actions and to exploit the similarity of instances of
a given type.

These definitions are adapted from \cite{tacas09tmphk}, with minor
restructuration and simplification. In particular, \emph{multisets} over an
alphabet of action labels are replaced by \emph{words} to simplify notations.

\subsection{ITS definition}

\textbf{ITS Type.}
The generic definition of an Instantiable Transition System (ITS) builds upon
the notion of model type and instance. It uses a composition mechanism based
solely on transition \emph{synchronization} (no explicit shared memory or
channel). Definition \ref{def:ipntype} sets an abstract contract or interface
that must be implemented by concrete ITS types.

\textbf{Notations:} The set of finite words over a set A (called
alphabet) is denoted by $\lang{A}$, with $\varepsilon$ for the empty
word and $\cdot$ for the concatenation operation.  We denote by $z.X,
z.Y \ldots$ the element $X$ (resp. $Y$\ldots) of a tuple $z =
\tuple{X,Y,\cdots}$.

\begin{definition}[ITS Semantics]
\label{def:ipntype}
An \textbf{ITS type} defines its semantics by a tuple\\
$\tau = \tuple{S,A,\locals{},\succs{}}$ where:
\begin{itemize}
\item $S$ is a set of states;
%$\istate{} \subseteq S$ is a finite set of initial states;
$A$ is a finite set of public action labels;
\item $\locals{}: S \mapsto 2^S$ is the local successors function;
\item $\succs{}: S \times \lang{A} \mapsto 2^S$ is the transition function
satisfying $\forall s \in S, \succs{}(s,\varepsilon)=\{s\}$.
\end{itemize}
Let $T$ be a set of ITS types. An \textbf{ITS instance} $i$ is defined
by its ITS type, denoted by $\mathit{type}(i) \in T$. An ITS instance
$i$ may be \emph{assigned} a state $s \in \mathit{type}(i).S$.

\textbf{Reachability}: A state $s'$ is reachable by an instance $i$
from the state $s_0$ iff $\exists s_1,\ldots s_n \in
\mathit{type(i)}.S$ such that $s' = s_n$  and $\forall 1 \leq i \leq n, s_i
\in \mathit{type(i)}.Locals(s_{i-1})$.

\end{definition}

%The set $\istate{}$ allows assigning of different initial states to different
%instances of a given type and 
$\locals{}$ typically returns states reachable
through occurrences of local events. $\locals{}$ represents actions that may
occur within an instance autonomously or independently from the rest of the
system.

The function $\succs{}$ produces successors by explicitly synchronizing
actions via a word over the alphabet of action labels. Synchronizing on an
empty word leaves the state of the instance locally unchanged. Note that
$\succs{}$ is the only way to control the behavior of a (sub)system from
outside.

Thus the transition relation of a full system can only be defined in terms of
synchronizations using $\succs{}$ and of independent local behaviors. A full
system is defined by an instance of a particular type in a specific initial
state: the system is self-contained and reachability only depends on the
definition of $\locals{}$.
